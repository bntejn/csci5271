% sections/experiment.tex
\begin{table}[]
\centering
\caption{Experiment design to answer {\it RQ1}}
\label{table:prefetch}
\begin{tabular}{lllll}
\cline{1-3}
\multicolumn{1}{|l|}{victim \textbackslash attacker} & \multicolumn{1}{l|}{prefetch on} & \multicolumn{1}{l|}{prefetch off} &  &  \\ \cline{1-3}
\multicolumn{1}{|l|}{prefetch on}                    & \multicolumn{1}{l|}{(1)}         & \multicolumn{1}{l|}{(2)}          &  &  \\ \cline{1-3}
\multicolumn{1}{|l|}{prefetch off}                   & \multicolumn{1}{l|}{(3)}         & \multicolumn{1}{l|}{(4)}          &  &  \\ \cline{1-3}
                                                     &                                  &                                   &  & 
\end{tabular}                  
\end{table}

\begin{enumerate}
\item
We speculate that prefetching itself might provide extra defense because of the extra packets.
It can also be the case however, prefetching websites are more vulnerable to fingerprinting because of the extra prefetch packets that shows distinct prefix.
\item
This case is unlikely since prefetching is on by default. We assume that victims will more likely be using Tor under the default setting.
\item
This case is what we are most curious about, whether a victim can confuse an attacker by simply turning prefetching setting off of his browser.
\item
This case simulates a situation where a victim is loading any other websites that does not prefetch any resource.
This can be used as a comparison case.
\end{enumerate}
