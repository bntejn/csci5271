% csci5271.tex
% this is the main LaTex file
\documentclass{sig-alternate-05-2015}

\begin{document}

\title{Investigating the Effects of Pre-Fetching on Website Fingerprinting Attack\titlenote{This report is submitted as a partial fulfillment of {\it CSCI5271: Introduction to Security} course.}}

\numberofauthors{4}
\author{
    \alignauthor
        Vaibhav Sharma\\%
        \email{sharm361@umn.edu}
    \alignauthor
        Taejoon Byun\\
        \email{taejoon@umn.edu}
    \alignauthor
        Se Eun Oh\\%
        \email{seoh@umn.edu}
    \alignauthor
        Elaheh Ghassabani\\
        \email{ghass013@umn.edu}
    \end{tabular}\newline\begin{tabular}{c}
    \affaddr{Department of Computer Science and Engineering}  \\
        \affaddr{University of Minnesota}   \\
        \affaddr{Minneapolis, MN 55454}
}

\maketitle
\begin{abstract}
\emph{This paper is written to get an A in the CSCI5271 course. PERIOD.}

\emph{This content will be edited later.} We plan to explore the area of website fingerprinting in anonymization networks starting with the paper Website Fingerprinting in Onion Routing Based Anonymization Networks.
\end{abstract}

% We no longer use \terms command
%\terms{Theory}

\keywords{Website fingerprinting, anonymity, encrypted traffic, Tor}

\section{Introduction}
% sections/intro.tex
Anonymizing networks are privacy technologies that provide a mancinism to anonymize internet communications so as to protect users from network eavesdroppers.
Although such systems are able to hide the communication (including both routing information and content), an attacker is still able to obtain different information by analyzing the network traffic.
Network analysis can provide very rich information about message length, timing, and frequency by which an attacker can easily identify the communicating parties, and therefore bypass an anonymizing system.
This problem is known as Website Fingerprinting (WF) attack, where an adversary attempts to recognize the encrypted traffic patterns of specific web pages without using any other information \cite{juarez14, murdoch2005low}.

% we should say, there is lots of work trying to prevent WF attack. And, in this project, we studied most of the state of the art works, and tries to provide a novel defend mechanism [or something similar]

% we want to add a summary about we have done

The remaining sections of this report is organized as follows: Section~\ref{sec2} provides a brief background about Tor anonymity network and website fingerprinting and Section~\ref{sec3} mentions related works on website fingerprinting attacks, defenses and criticisms.
We suggest link prefetching as a defense mechanism in Section~\ref{sec4} followed by experimental evaluation (Section~\ref{sec5})
Section~\ref{sec6} discusses about the feasibility of the proposed defense based on experimental result, and finally conclude our report on Section~\ref{sec7}.



\section{Related Work}
% sections/related.tex
% 3. Related Works

This section surveys attacks/ defenses on/against anonymizing systems, especially Tor. 

\subsection{Fingerprinting Attacks}

The idea of inferring meaningful information based on sniffing and analyzing encrypted SSL packets, which is called deep packet inspection (DPI), was introduced in 1996 \cite{wagner96}.
Later in 2002, Hintz presented the first website fingerprinting attack on an encrypted web proxy.
The attack shows how DPI is powerful enough to identify a specific web site that the user is surfing \cite{hintz2003}.
Since then, a lot of research has been conducted in this area trying either to propose a more realistic WF attack, or a more effective defense mechanism against WF.

Herrmann et al. \cite{herrmann2009} pointed out that widely used anonymous networks such as Tor still failed to defend against WF attack.
They achieved only a 3\% success rate for 775 pages using Multinomial Naive Bayes classifier focusing only on packet sizes.
The WF attacks on Tor suggested by Penchenko et al. \cite{panchenko11} showed more reasonable accuracy with more sophisticated machine learning techniques and diverse features.
Later on, based on \cite{panchenko11}'s attack model, many works such as \cite{wang2013improved, cai2012touching} tried to build more powerful and realistic attacks that improved the accuracy rate up to around 90\% in closed word setting.
They did thorough inverstigation on extracting powerful features such as Tor cells and improving classification method.
Especially, Cai et al. \cite{cai2012touching} had shown that their novel WF attack successfully defeated existing well-known defense mechanisms such as HTTPOS \cite{luo2011}.
However, even though existing works have successfully improved the effectiveness of WF attacks on Tor, underlying assumptions for experiments significantly simplified real world setting.
As Juarez et al. \cite{juarez14} pointed out that existing WF attacks are unrealistic with assumptions such as the client setting, we decided to focus on building more robust and practical defense mechanisim instead of developing advanced attack models to improve the accuracy rate.
%%%TODO \textbf{please add  or modify the comparison of our work to existing works}

%Generally speaking, attacks on anonymizing networks take a variety of approaches; some of the attacks tries to discover the identity of the anonymous user, others focus on uncovering the private pathway, and others attempt to identify the servers users interact with  \cite{cai2012touching}.


\subsection{Defense Mechanisms}

Defense mechanisms are usually developed at the IP/TCP level by changing the pattern of traffic.
For example, they splited packets into multiple packets, injected spurious packets into the traffic, or involved  padding packets to prevent from leaking features of traffic.
A study performed in \cite{fu2003} suggested to transmit packets at random intervals as a defense mechanism against traffic analysis.
Another defense technique proposed in \cite{wright2009}, called traffic morphing, uses some tricks to change a traffic pattern to disguise new or existing other traffic.
However, since their approach still leaks the information such as packet order, attacks that work without packet size information can easily defeat their mechanism.

In \cite{luo2011}, Luo et al. proposed novel HTTP/TCP level defense mechanisms against some analysis attacks by changing window sizes and the order of packets in the TCP stream.
Besides, at the HTTP level, they tried to inject some extra data into HTTP GET headers, generate some irrelevant HTTP request, and re-order requests.
The Tor community also introduced "randomized pipelining" \cite{perry11} to defend the WF, by having the browser load the web content in a random order.
Despite such techniques, the proposed attack in \cite{cai2012touching} managed to successfully recognize target web pages with the accuracy of 87\%. 
Howerver, all exisitng defenses have not been neither efficient nor effective for fancy WF attacks. %This attack is able to ignore packet sizes, while most of the attacks on Tor work based on packet sizes.
In contrast, our work more focuses on utilizing existing HTML systax that is link prefetching shown in Section 4, which subsequently does not accomodate extra cost.

\subsection{Criticism}

In the literature, some work always debate over the practical feasibility of the WF attack. This section tries to summarize issues questioning the practicality of WF.

There are many different factors that play a role in the success of WF.
The main criticism is that academic papers usually oversimplify WF attack models by making some unrealistic assumptions over users' browsing habits, training and testing traces, even the version of Tor used for testing/ training.
In \cite{juarez14}, it is discussed that the studies performed in \cite{cai2012touching, herrmann2009, panchenko11, wang2013improved, shi2009} simplify the problem and overestimate the adversary's capabilities.

Another criticism is that all works lunch attacks on individual pages, instead of overall websites.
So, although the attack is called website fingerprinting, it is actually about webpage fingerprinting.
In addition, there are some main factors, including the the hypothesis state space and the size of the instance space, that affect the accuracy of a classifier.
Even, the number of training samples provided to the classifier and false positive rates matters.
Since reliable feature information is constant, with the increase in the number of classification categories, the classifier eventually runs out of descriptive feature information, which causes either true positive accuracy goes down or the false positive rate goes up.
A Tor blog article~\cite{TorBlog} discusses that the effects of such factors are quite observable in the papers with a sufficient world size.

Although more attention should be paid to the scenarios by which attacks are evaluated, we should not dismiss WF as a threat. However, \cite{TorBlog} argues that, due to theoretical and practical issues, realistic WF attacks are hard to lunch on Tor. Therefore, it claims that even simple defenses could protect Tor users against WF. It seems that the Tor community believes that defense mechanisms do not need to be very complicated to be effective.

To the best of our knowledge, none of the defense techniques has investigated the effect of link pre-fecting on WF. In the next section, we will describe the concept of link pre-fetching. Since most of the defenses are applied at the Tor network, they are required to be acceptable by the Tor community. However, we aim to propose a defense technique which can also be used by the website owners. 
\subsection{Web Prefetching}
Web prefetching has been widely used to reduce the latency for users. Fan et al.\cite{Fan1999} introduced the first prefetching between caching proxies and clients. Chen et al. \cite{Chen2003} pre-fetched web pages that will be visited near future. In addition, there have been several works using prefetching over anonymous network. Nguyen et al. \cite{} prefetched web components of web page, which users are visiting, to solve the delay performance of Tor. They set two proxies, one sits between the client and browser, and the other sits between the exit relay and the website. Even though they successfully improved tor performance, the extra setting of proxies and the communication over them are not good in terms of security since we need to trust them. (e.g., proxies are compromised by adversaries). In contrast, we do not enforce any additional setting on tor network as well as client-side to hinder from such additional security risks.  \textbf{please add  or modify the comparison of our work to existing works}




\section{Background}
% sections/background.tex

This section provides a brief description of required background.

\subsection(Website Fingerprinting}

\subsection{Link Pre-fetching}

Link prefetching is a HTML syntax that gives the web browser a hint about which page the user is most likely to visit in the near future.
For the prefetching resources that a web page specifies, the browser silently loads them after idle time and store them in cache. 
It was first suggested by Mozilla Foundation, and is adopted by most modern browsers nowadays.

Today's web browsers makes use of a specific syntax called \emph{pre-fetching}, which was proposed as a draft standard by Mozilla.
Using pre-fetching, browser can predicts documents likely to be visited by the user in the near future.
Therefore, based on the hint provided by pre-fetching a browser is able to fetch those documents a head of time.
In fact, it is the web page that provides a set of pre-fetching hints for the browser.
Then, loading the page and passing an idle time, the browser starts to pre-fetch and cache specified documents.
Needless to say, this mechanism improves efficiency.
Particularly, it is most effective if the content provider may be reasonably certain which links users are going to visit next \cite{wikiPreF}.



\section{Effects of Pre-Fetching on Fingerprinting}
In this section, we will write  about our experiments. We are planning to conduct two sets of experiments. If we consider the network traffic, the number of packages go upstream depends on the number of pre-fetching requests, and the number of downstream packages coming depends on the size of resources that should be pre-fetched. Therefore, it is obvious that pre-fetching would affect the fingerprint of the traffic of a particular website.  \emph{should be completed.}

\subsection{Investigate Pre-Fetching Effects on top 60 Popular Websites}
We are running experiments to see how pre-fetching affects the websites' fingerprints.
After doing some search on top popular websites, we put together a small crawler by which we learnt that only around 60 of all 6000 websites are use pre-fetching mechanism. We are capturing traffic of these websites in two different modes: 1) with enabled pre-fetching, and 2) with disabled pre-fetching. We are working on feature extraction, and about to decide which classifiers to use for the learning phase.
Ultimately, we plan to conduct two sets of experiments. One sort of experiment is to compare two series of the captured packets and find the accuracy number with the help of a classifier, by which our goal is to provide an evidence to see if pre-fetching really affects fingerprints of websites. So, if the result will be positive, we will perform another set of experiment, which kind of simulates a sub set of those 60 websites. Then, we will see how (altering) the size of pre-fetching affects fingerprinting attacks/ defense mechanisms.


\subsection{Effect of Pre-Fetching Packets Size on Fingerprinting Attacks}
Here, we will explain our second experiment. We will simulate a sub set of webpages we investigated in the previous experiment. Then, we will equip them with a mechanism so that they can affect the downstream traffic and finally their fingerprint. Then, we will analyze the result to see how this idea contributes to the effectiveness of attacks and defense techniques.

\section{Experiments}
% sections/experiment.tex

\section{Research Question}

\section{Effects of Pre-Fetching on Fingerprinting}
In this section, we will write  about our experiments. We are planning to conduct two sets of experiments. If we consider the network traffic, the number of packages go upstream depends on the number of pre-fetching requests, and the number of downstream packages coming depends on the size of resources that should be pre-fetched. Therefore, it is obvious that pre-fetching would affect the fingerprint of the traffic of a particular website.  \emph{should be completed.}

\subsection{Investigate Pre-Fetching Effects on top 60 Popular Websites}
We are running experiments to see how pre-fetching affects the websites' fingerprints.
After doing some search on top popular websites, we put together a small crawler by which we learnt that only around 60 of all 6000 websites are use pre-fetching mechanism. We are capturing traffic of these websites in two different modes: 1) with enabled pre-fetching, and 2) with disabled pre-fetching. We are working on feature extraction, and about to decide which classifiers to use for the learning phase.
Ultimately, we plan to conduct two sets of experiments. One sort of experiment is to compare two series of the captured packets and find the accuracy number with the help of a classifier, by which our goal is to provide an evidence to see if pre-fetching really affects fingerprints of websites. So, if the result will be positive, we will perform another set of experiment, which kind of simulates a sub set of those 60 websites. Then, we will see how (altering) the size of pre-fetching affects fingerprinting attacks/ defense mechanisms.


\subsection{Effect of Pre-Fetching Packets Size on Fingerprinting Attacks}
Here, we will explain our second experiment. We will simulate a sub set of webpages we investigated in the previous experiment. Then, we will equip them with a mechanism so that they can affect the downstream traffic and finally their fingerprint. Then, we will analyze the result to see how this idea contributes to the effectiveness of attacks and defense techniques.


Research Questions
\begin{enumerate}
\item
{\bf RQ1}: Does prefetching itself provide an extra degree of defense?
\item
{\bf RQ2}: Can prefetching be used as a browser-side defense mechanism?
\end{enumerate}

\begin{table}[]
\centering
\caption{Experiment design to answer {\it RQ1}}
\label{table:prefetch}
\begin{tabular}{lllll}
\cline{1-3}
\multicolumn{1}{|l|}{victim \textbackslash attacker} & \multicolumn{1}{l|}{prefetch on} & \multicolumn{1}{l|}{prefetch off} &  &  \\ \cline{1-3}
\multicolumn{1}{|l|}{prefetch on}                    & \multicolumn{1}{l|}{(1)}         & \multicolumn{1}{l|}{(2)}          &  &  \\ \cline{1-3}
\multicolumn{1}{|l|}{prefetch off}                   & \multicolumn{1}{l|}{(3)}         & \multicolumn{1}{l|}{(4)}          &  &  \\ \cline{1-3}
                                                     &                                  &                                   &  & 
\end{tabular}                  
\end{table}

\begin{enumerate}
\item
We speculate that prefetching itself might provide extra defense because of the extra packets.
It can also be the case however, prefetching websites are more vulnerable to fingerprinting because of the extra prefetch packets that shows distinct prefix.
\item
This case is unlikely since prefetching is on by default. We assume that victims will more likely be using Tor under the default setting.
\item
This case is what we are most curious about, whether a victim can confuse an attacker by simply turning prefetching setting off of his browser.
\item
This case simulates a situation where a victim is loading any other websites that does not prefetch any resource.
This can be used as a comparison case.
\end{enumerate}


\section{Conclusions}
% sections/conclusion.tex

%What we have done in this work is so cool and awesome.
%What you may criticize will all be put here as ''future work''.

%This section will conclude the result of our experiments. Finally we will provide some evidence to show how pre-fetching affect fingerprinting attacks. Based on our result, we are planing to suggest some defense mechanisms.
We proposed a new method for defending against website fingerprinting attacks on the Tor anonymity network. 
Link prefetching allows web pages to specify resources which are expected to be downloaded by the browser in the near future and provides web developers with an opportunity to improve user experience on their websites. 
We presented how link prefetching can be useful not only for improved browsing experience but also for acting as a defense mechanism against fingerprinting attacks. 
We showed the effect of link prefetching on different features used by fingerprinting attack classifiers. 
We formulated the challenge of using link prefetching as a defense mechanism in the form of two research questions and designed experiments to evaluate answers to our research questions. 
We create three different classifiers to evaluate the effect of the link prefetching setting on existing real world link prefetching performed by 60 of the 6000 most popular Alexa websites and found the existence of prefetching traffic to reduce the accuracy of classifiers.
We then tried to evaluate the effect of changing the parameter values on the fingerprintability of a webpage but found that the Tor browser does not honor the prefetching requests specified in our test webpage. This behavior deviates from the expected behavior of downloading the prefetching requests once the webpage itself has finished downloading. Finally we discussed how a webpage can calculate the value of the number of prefetching requests and size of prefetched responses in order to disguise its fingerprint. 



\section{Acknowledgments}
The authors appreciate Professor Stephen McCamant for telling geeky jokes in classes all the time.
This is a research project for CSCI5271, University of Minnesota.

\bibliographystyle{abbrv}
\bibliography{csci5271} 

\end{document}
