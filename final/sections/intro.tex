% sections/intro.tex
%%%%%%%%%%%%%%%%%%%%%%%%%%%%%%%%%%%%%%%%%%%%%%%%%%%%%%%%%%%%%%%%%%%%%%%%%%%%%%%
% The structure of introduction
% 1. Tor & WF
% 2. Existing WF attacks
% 3. Existing defenses
% 4. Our proposed approach - link prefetching
% 5. Experimental evaluation
% 6. Contribution
% 7. Remaining sections
%%%%%%%%%%%%%%%%%%%%%%%%%%%%%%%%%%%%%%%%%%%%%%%%%%%%%%%%%%%%%%%%%%%%%%%%%%%%%%%

% 1. Tor & WF 
Tor is an overlay network that provides communication anonymity and security using onion routing.
An onion network encapsulates a message in layers of encryption and reroutes the messages throughout volunteering nodes, so as to hide not only the contents of the message but also the location of the origin.
Although the main goal of Tor is to protect its users against {\it traffic analysis}, which is a form of Internet surveillance, it is still vulnerable to a targeted attack called website fingerprinting (WF).
An adversary of WF attempts to infer which website a victim is visiting, and this can be a serious threat in privacy-sensitive use cases.

% 2. Existing WF attacks
Website fingerprinting gained attention since the early work by Hermann et al.~\cite{hermann}, when the authors achieved 3\% of success rate using a Naive Bayes classifier.
Although the success rate is too low to be practical, following works using a more sophisticated classification techniques achieved a much higher accuracy around 90\% \cite{panchenko11, wang2013improved, cai2012touching}, showing that website fingerprinting can be a serious threat against the users of the anonymity network.
Especially, Cai et al.~\cite{cai2012touching} showed that their attack can successfully defeat existing well-known defense mechanisms such as HTTPOS~\cite{luo2011}.

% 3. Defense
Several defenses were also suggested accordingly to defend users against fingerprinting attacks.
Those defense mechanisms are usually developed at the IP/TCP level by changing the pattern of the traffic whether by splitting, morphing, injecting, or padding packets~\cite{fu2003, wright2009, luo2011, perry11}.
Many of the defense mechanisms, however, had been neutralized by following attacks~\cite{cai2012touching}.
Moreover, all of the defenses involve making changes either to the Tor protocol itself or to the browser. Using link prefetching as a defense mechanism has the distinct advantage of not requiring any changes to be made to the protocol implementation or the browser. It provides more browser-side control which allows this defense to be deployed as a browser extension. 

% 4. Why prefetching
In this work, we propose a novel technique to provide defense against WF attacks using link prefetching.
Link prefetching is a HTML5 feature that gives hints to a browser the list of resources and pages to prefetch, when it can be sure about which page a user is likely to request next.
When the browser encounters prefetch tags, it silently loads and caches the specified resources after the page is fully loaded.
Since prefetching generates extra outgoing and incoming packets, it affects the features of website fingerprint and thus undermines the accuracy of the classifiers used in WF attack (Section~\ref{sec4}).

% 5. Experimental validation
We validated our approach through a series of experiments for the two research questions -- (1) {\it does prefetching itself provide an extra degree of defense?} (2) {\it can prefetching be uses as a browser-side mechanism?}

% we should say, there is lots of work trying to prevent WF attack. And, in this project, we studied most of the state of the art works, and tries to provide a novel defend mechanism [or something similar]
% we want to add a summary about we have done

% 6. Contribution
The major contribution of our work is twofolds:
\begin{itemize}
\item
We present a novel approach of using link prefetching as a defense mechanism against WF attacks.
\end{itemize}

% 8. Remaining sections
The remaining sections of this report is organized as follows: Section~\ref{sec2} provides a brief background about Tor anonymity network and website fingerprinting, and Section~\ref{sec3} mentions related works on website fingerprinting attacks, defenses and criticisms.
We suggest link prefetching as a defense mechanism in Section~\ref{sec4} followed by experimental evaluation (Section~\ref{sec5}).
Section~\ref{sec6} discusses about the feasibility of the proposed defense based on experimental result, and finally conclude our report on Section~\ref{sec7}.

