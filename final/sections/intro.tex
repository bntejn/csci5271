% sections/intro.tex
%%%%%%%%%%%%%%%%%%%%%%%%%%%%%%%%%%%%%%%%%%%%%%%%%%%%%%%%%%%%%%%%%%%%%%%%%%%%%%%
% The structure of introduction
% 1. Tor & WF
% 2. Why is WF serious
% 3. Existing WF attacks
% 4. Existing defenses
% 5. Our proposed approach - link prefetching
% 6. Experimental evaluation
% 7. Contribution
% 8. Remaining sections
%%%%%%%%%%%%%%%%%%%%%%%%%%%%%%%%%%%%%%%%%%%%%%%%%%%%%%%%%%%%%%%%%%%%%%%%%%%%%%%

\iffalse
% 1. Tor & WF (previously written by Elaheh)
Anonymizing networks are privacy technologies that provide a mancinism to anonymize internet communications so as to protect users from network eavesdroppers.
Although such systems are able to hide the communication (including both routing information and content), an attacker is still able to obtain different information by analyzing the network traffic.
Network analysis can provide very rich information about message length, timing, and frequency by which an attacker can easily identify the communicating parties, and therefore bypass an anonymizing system.
This problem is known as Website Fingerprinting (WF) attack, where an adversary attempts to recognize the encrypted traffic patterns of specific web pages without using any other information \cite{juarez14, murdoch2005low}.
\fi

% 1. Tor & WF (TJ)
Tor is an overlay network that provides communication anonymity and security using onion routing.
An onion network encapsulates a message in layers of encryption and reroutes the messages throughout volunteering nodes, so as to hide not only the contents of the message but also the location of the origin.
Although the main goal of Tor is to protect its users against {\it traffic analysis}, which is a form of Internet surveillance, it is still vulnerable to a targeted attack called website fingerprinting (WF).
An adversary of WF attempts to infer which website a victim is visiting, and this can be a serious threat in privacy-sensitive use cases.

% 2. Existing WF attacks
Website fingerprinting gained attention since the early work by Hermann et al.~\cite{hermann2009}, when the authors achieved 3\% of success rate using a Naive Bayes classifier.
Although the success rate is too low to be practical, following works using a more sophisticated classification techniques achieved a much higher accuracy around 90\%~\cite{panchenko11, wang2013improved, cai2012touching}, showing that website fingerprinting can be a serious threat against the users of the anonymity network.
Especially, Cai et al.~\cite{cai2012touching} showed that their attack can successfully defeat existing well-known defense mechanisms such as HTTPOS~\cite{luo2011}.

% 3. Defense
Several defenses were also suggested accordingly to defend users against fingerprinting attacks.
Those defense mechanisms are usually developed at the IP/TCP level by changing the pattern of traffic, whether by splitting, morphing, injecting, or padding packets~\cite{luo2011, }.

For example, they splited packets into multiple packets, injected spurious packets into the traffic, or involved  padding packets to prevent from leaking features of traffic.
A study performed in \cite{fu2003} suggested to transmit packets at random intervals as a defense mechanism against traffic analysis.
Another defense technique proposed in \cite{wright2009}, called traffic morphing, uses some tricks to change a traffic pattern to disguise new or existing other traffic.
However, since their approach still leaks the information such as packet order, attacks that work without packet size information can easily defeat their mechanism.

In \cite{luo2011}, Luo et al. proposed novel HTTP/TCP level defense mechanisms against some analysis attacks by changing window sizes and the order of packets in the TCP stream.
Besides, at the HTTP level, they tried to inject some extra data into HTTP GET headers, generate some irrelevant HTTP request, and re-order requests.
The Tor community also introduced "randomized pipelining" \cite{perry11} to defend the WF, by having the browser load the web content in a random order.
Despite such techniques, the proposed attack in \cite{cai2012touching} managed to successfully recognize target web pages with the accuracy of 87\%. 
Howerver, all exisitng defenses have not been neither efficient nor effective for fancy WF attacks. %This attack is able to ignore packet sizes, while most of the attacks on Tor work based on packet sizes.
In contrast, our work more focuses on utilizing existing HTML systax that is link prefetching shown in Section 4, which subsequently does not accomodate extra cost.


% we should say, there is lots of work trying to prevent WF attack. And, in this project, we studied most of the state of the art works, and tries to provide a novel defend mechanism [or something similar]
% we want to add a summary about we have done

% 7. Contribution
The major contribution of our work is twofolds:
\begin{itemize}
\item
We present a novel approach of using link prefetching as a defense mechanism against WF attacks.
\end{itemize}

% 8. Remaining sections
The remaining sections of this report is organized as follows: Section~\ref{sec2} provides a brief background about Tor anonymity network and website fingerprinting and Section~\ref{sec3} mentions related works on website fingerprinting attacks, defenses and criticisms.
We suggest link prefetching as a defense mechanism in Section~\ref{sec4} followed by experimental evaluation (Section~\ref{sec5})
Section~\ref{sec6} discusses about the feasibility of the proposed defense based on experimental result, and finally conclude our report on Section~\ref{sec7}.

